%%=============================================================================
%% Gebruikte technologieën
%%=============================================================================

\chapter{Gebruikte technologieën}
\label{ch:technologieën}

%% TODO: Hoe ben je te werk gegaan? Verdeel je onderzoek in grote fasen, en
%% licht in elke fase toe welke stappen je gevolgd hebt. Verantwoord waarom je
%% op deze manier te werk gegaan bent. Je moet kunnen aantonen dat je de best
%% mogelijke manier toegepast hebt om een antwoord te vinden op de
%% onderzoeksvraag.


\section{Spring en Spring Boot}
\label{sec:spring-boot}

Dit onderzoek maakt gebruik van Spring Boot om microservices op te stellen. Het Spring framework staat bekend voor zijn Inversion of Control (IoC) principe, ook gekend als dependency injection. Dit betekent onder andere dat een object geannoteerd kan worden met \texttt{@Component} en Spring zal het object aanmaken, alle nodige velden vullen en het object toevoegen aan de context van de applicatie. Op deze manier is het mogelijk om verschillende objecten toe te voegen aan de context, zodat ze onderling gemakkelijk samenwerken. Het IoC principe vergemakklijkt hierdoor de manier om andere bibliotheken te integreren in de applicatie. \\

Spring Boot maakt het gemakkelijk om op zichzelf staande, productiewaardige Spring applicaties te maken waar het niet nodig is om veel aan de configuratie te sleutelen. Het mantra van Spring Boot is dan ook \textit{convention-over-configuration}. Dit is bijzonder handig voor microservices omdat bijvoorbeeld een eenvoudige REST applicatie aangemaakt kan worden in minder dan 20 lijnen code. \\

\begin{lstlisting}[language=Java, caption=eenvoudige Spring Boot REST app]
package hello;

import org.springframework.boot.*;
import org.springframework.boot.autoconfigure.*;
import org.springframework.stereotype.*;
import org.springframework.web.bind.annotation.*;

@Controller
@EnableAutoConfiguration
public class SampleController {

    @RequestMapping("/")
    @ResponseBody
    String home() {
        return "Hello World!";
    }

    public static void main(String[] args) throws Exception {
        SpringApplication.run(SampleController.class, args);
    }
}
\end{lstlisting}

De \texttt{@Controller} annotatie zorgt ervoor dat de klasse gebruikt kan worden door Spring MVC om web requests te behandelen. Vanaf Spring 4.0 is het ook mogelijk om \texttt{@RestController} te gebruiken, die de annotaties \texttt{@Controller} en \texttt{@ResponseBody} combineert. De laatste annotatie is handig om \texttt{return} waarden van requests op te vangen in een zelfgedefinieerde Java klasse. \texttt{@EnableAutoConfiguration} zegt dat Spring Boot de applicatie automatisch moet configureren op basis van de toegevoegde bibliotheken. Ten slotte is er nog de \texttt{@RequestMapping} annotatie die Spring vertelt dat het pad \texttt{/} gemapped moet worden op de \texttt{home} methode. \\

\section{Containerisatie met Docker}
\label{sec:docker}

Elke microservice, bijvoorbeeld een Spring Boot applicatie, kan door Docker uitgevoerd worden zijn eigen container. De verschillende containers zijn geïsoleerd van elkaar en delen enkel de minimale kernel van het besturingssysteem. Containers lijken op het eerste zicht op virtuele machines, in het opzicht dat beide geïsoleerde omgevingen zijn die beheerd worden door een controlerend proces: een container manager en hypervisor respectievelijk. Het grootste verschil tussen de twee is echter dat, voor elke virtuele machine, een volledige stack van componenten uitgevoerd dienen te worden: het besturingssysteem tot en met de applicatielaag en de virtuele hardware met netwerkkaarten, CPU's en geheugen. \\

Containers daarentegen functioneren eerder als volledig geïsoleerde \textit{sandboxes}. De containers delen met elkaar de kernel van het besturingssysteem met de aanwezige systeembronnen. Dit betekent dat containers veel minder zwaar zijn op het onderliggende systeem, zodat meer containers uitgevoerd kunnen worden dan virtuele machines. Een belangrijke limitatie van containers is echter dat ze enkel uitgevoerd kunnen worden in Linux-gebaseerde besturingssystemen. Dit omdat Docker gebruik maakt van kernel isolatie, wat een specifieke Linux technologie is. \\

Docker kan niet rechtstreeks uitgevoerd worden op Mac of Windows systemen, maar er is een eenvoudige workaround om dit te verhelpen: Docker Toolbox. Er zal eerst een Linux virtuele machine opgestart worden in VirtualBox, waarna de docker containers uitgevoerd kunnen worden in deze virtuele machine. \\

\section{Tracing}
\label{sec:tracing}

Een van de uitdagingen in een microservice omgeving is om het overzicht te bewaren hoe alle verschillende services met elkaar samenwerken. Omdat elke service onafhankelijk is van de volgende, is het moeilijker om het gedrag van het volledige systeem te controleren. Zoals eerder vermeld in \ref{ch:inleiding} kan tracing hierbij helpen en omdat het gaat over tracing in een microservice systeem, spreekt men ook over distributed tracing. \\

\subsection{Spring Cloud Sleuth}
\label{sec:sleuth}

Spring Cloud biedt een aantal bibliotheken aan zoals Config (een plaats om configuraties van verschillende microservices te beheren), Netflix (Netflix OSS integraties voor Spring Boot apps zoals Eureka) en dus ook Sleuth. Sleuth is een distributed tracing oplossing die concepten van Dapper, Zipkin en HTrace leent. \\

Een trace ziet er bij Sleuth uit zoals bij Dapper. Het bevat een trace id dat gevormd wordt wanneer de eerste request gemaakt wordt. Voor elke service die de request doorkruist, wordt een span id toegekend voor die service en toegevoegd aan de trace. Sleuth voegt die id's toe aan de headers in de request response. (invoegen voorbeeld trace, met trace id en span id headers) De eerste span van een trace kent dezelfde id als de trace. \\

Sleuth voegt de trace en span id's ook toe aan de logging voor de microservice. Dit ziet er zo uit: \texttt{[my-service-id,73b62c0f90d11e06,73b62c0f90d11e06,false]}. \texttt{my-service-id} is de naam van de service, de eerste id duidt de trace aan, de volgende is de span en de laatste waarde geeft aan of de span geëxporteerd dient te worden naar Zipkin. \\

\subsection{Zipkin}
\label{sec:zipkin}

Om de traces die Sleuth aanbiedt te verzamelen en te analyseren, komt Zipkin goed van pas. Sleuth heeft de mogelijkheid om tracing informatie te sturen naar een Zipkin server door de dependency \texttt{spring-cloud-sleuth-zipkin} toe te voegen aan de applicatie. Sleuth gaat er standaard vanuit dat de Zipkin server loopt op \texttt{http://localhost:9411}. De locatie kan echter aangepast worden door \texttt{spring.zipkin.baseUrl} toe te voegen aan de applicatie eigenschappen (application.properties). \\


