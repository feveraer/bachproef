%%=============================================================================
%% Inleiding
%%=============================================================================

\chapter{Inleiding}
\label{ch:inleiding}

Monolithische architecturen ruimen steeds meer baan voor microservice architecturen of kortweg microservices. Een monolitische applicatie is gebouwd als een enkelvoudige, zelfstandige eenheid. Bij een client-server model, kan bijvoorbeeld de applicatie langs de server zijde bestaan uit een enkele applicatie die de HTTP requests behandelt, logica uitvoert en data ophaalt of bijwerkt in de database. Het grote probleem van monolitische applicaties is dat ze moeilijk te onderhouden zijn. Een kleine verandering in een bepaalde functionaliteit van de applicatie kan ervoor zorgen dat andere delen van de applicatie ook bijgewerkt moeten worden. De volledige applicatie moet opnieuw gebuild en gedeployed worden. Als een bepaalde functionaliteit gescaled moet worden, moet de volledige applicatie gescaled worden. Microservices bieden hiervoor een oplossing, aangezien de verschillende functionaliteiten elk een op zich zelf staande service kunnen vormen.

Bij microservices schieten echter de traditionele tools voor debugging tekort. Een enkele service kan niet het volledige beeld geven over bijvoorbeeld de performantie van de applicatie in zijn geheel. Om dit te verhelpen, kunnen requests getraced worden. Een trace is de volledige reisweg van een request die spans bevat voor alle doorkruiste microservices. Een span bestaat uit tags of metadata zoals de start- en stoptijdstippen. Deze data kan dan verzameld worden om een volledig beeld van het gedrag van de applicatie te geven.

De bedoeling van dit onderzoek is om de meerwaarde van tracing in microservices aan te tonen en ook om eventuele tekortkomingen vast te stellen. Er wordt specifiek gekeken naar tracing van requests met behulp van Spring Cloud Sleuth en Zipkin. In de praktijk gaat men niet alle requests gaan tracen om onnodige overhead te vermijden. Beslissen welke requests getraced worden en welke niet wordt sampling genoemd. Wat kunnen bijvoorbeeld enkele interessante use cases zijn waarbij er op een intelligente manier gesampled kan worden? Ten slotte visualiseert Zipkin enkel de verschillende traces, maar niet de log berichten, deze worden dan ook nog niet verzameld. Om logs te verzamelen en visualiseren wordt gekeken naar ElasticSearch, Logstash / Fluentd en Kibana. Er wordt aangetoond hoe deze tools in combinatie met Zipkin helpen om de complexiteit van microservices in bedwang te houden.

\section{Stand van zaken}
\label{sec:stand-van-zaken}

%% deze sectie (die je kan opsplitsen in verschillende secties) bevat je
%% literatuurstudie. Vergeet niet telkens je bronnen te vermelden!

Onderzoeken rond distributed tracing systemen zijn schaars, maar de technologieën die gebruikt worden in dit onderzoek, namelijk Spring Cloud Sleuth en Zipkin, zijn gebaseerd op Google Dapper.~\autocite{Sigelman2010} Deze technische paper beschrijft in detail de werking van Google's distributed tracing systeem. Net zoals Dapper gebruikt Sleuth een annotatie gebaseerde methode om tracing toe te voegen aan requests. De terminologie is volledig overgenomen. Een trace bevat meerdere spans voor elke hop naar een andere service. Spans die dezelfde trace id bevatten, maken deel uit van dezelfde trace. Een groot ontwerpdoel van Dapper was om de overhead voor tracing zo laag mogelijk te houden. Sampling werd hierdoor geïntroduceerd. In een eerste productie gebruikten ze eenzelfde sampling percentage voor alle processen bij Google, gemiddeld één trace per 1024 kandidaten. Dit schema bleek heel effectief te zijn voor diensten met veel verkeer, maar bij diensten met minder verkeer gingen er zo interessante events verloren. Als oplossing werd toegelaten dat dit sampling percentage wel aan te passen was, ook al wou Google met Dapper dit soort manuele interventie vermijden. Spring Cloud Sleuth neemt dezelfde instelling over en laat ook toe om zelf sampling in te stellen.

\section{Probleemstelling en Onderzoeksvragen}
\label{sec:onderzoeksvragen}

Het is voorlopig nog onduidelijk welke performantiegevolgen tracing met zich meedraagt in een Spring Boot microservices omgeving. Deze worden in dit onderzoek bekeken door de sampling van Sleuth in te stellen.

Om logs van een Spring Boot microservice naar ElasticSearch te sturen worden Logstash en Fluentd bekeken. Is er een performantieverschil?

%Fluent logger Fluency zou 4x sneller zijn dan de standaard java fluent-logger. Is dit altijd zo?

%% TODO:
%% Uit je probleemstelling moet duidelijk zijn dat je onderzoek een meerwaarde
%% heeft voor een concrete doelgroep (bv. een bedrijf).
%%
%% Wees zo concreet mogelijk bij het formuleren van je
%% onderzoeksvra(a)g(en). Een onderzoeksvraag is trouwens iets waar nog
%% niemand op dit moment een antwoord heeft (voor zover je kan nagaan).

\section{Opzet van deze bachelorproef}
\label{sec:opzet-bachelorproef}

%% TODO: Het is gebruikelijk aan het einde van de inleiding een overzicht te
%% geven van de opbouw van de rest van de tekst. Deze sectie bevat al een aanzet
%% die je kan aanvullen/aanpassen in functie van je eigen tekst.

De rest van deze bachelorproef is als volgt opgebouwd:

In Hoofdstuk~\ref{ch:technologie} worden de gebruikte technologieën in dit onderzoek besproken.
In Hoofdstuk~\ref{ch:onderzoek} wordt de methodologie toegelicht en worden antwoorden gezocht op de onderzoeksvragen.
In Hoofdstuk~\ref{ch:conclusie}, wordt tenslotte de conclusie gegeven en een antwoord geformuleerd op de onderzoeksvragen.

