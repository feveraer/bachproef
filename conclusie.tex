%%=============================================================================
%% Conclusie
%%=============================================================================

\chapter{Conclusie}
\label{ch:conclusie}

%% TODO: Trek een duidelijke conclusie, in de vorm van een antwoord op de
%% onderzoeksvra(a)g(en). Wat was jouw bijdrage aan het onderzoeksdomein en
%% hoe biedt dit meerwaarde aan het vakgebied/doelgroep? Reflecteer kritisch
%% over het resultaat. Had je deze uitkomst verwacht? Zijn er zaken die nog
%% niet duidelijk zijn? Heeft het ondezoek geleid tot nieuwe vragen die
%% uitnodigen tot verder onderzoek?

Voor dit onderzoek was het onduidelijk wat de performantiegevolgen van tracing met Sleuth en Zipkin zouden zijn. Er kan nu gesteld worden dat tracing wel degelijk het systeem extra belast, maar dit enkel merkbaar zal zijn in een microservice architectuur die minstens zo complex is als in figuur \ref{fig:hc_dependencies}. In zo een opstelling werd bij hoog netwerkverkeer (minstens 1000 tegelijke requests) een verdubbeling van de responstijd genoteerd. Belangrijk is dat die verdubbeling in responstijd pas optreedt vanaf een 50\% sampling frequentie. Door het drukken van de sampling frequentie kan de performantielast, die tracing met zich meedraagt, geminimaliseerd worden. Bij een 10\% sampling frequentie is de responstijd terug normaal. Dit ligt in de lijn met de verwachtingen voor dit onderzoek. In de Dapper paper~\autocite{Sigelman2010} werd immers genoteerd dat het niet nodig is om alle requests te tracen in een systeem dat gebukt gaat onder een hoog netwerkverkeer. Als er immers een probleem zou opduiken in het systeem, zal dat probleem zich herhalen in meerdere requests en uiteindelijk zal minstens één van die requests getraced worden.

Het was ook nog onduidelijk voor dit onderzoek of het verzamelen van de log berichten van de microservices extra performantiegevolgen hadden op het systeem. Er kan nu gesteld worden dat dit, in alle opstellingen met verschillend netwerkverkeer, geen grote invloed heeft op de totale performantie van het systeem. Opmerkelijk is dat Fluentd beter presteerde als Logstash in dit onderzoek. Er kan besloten worden dat, in een microservices opstelling met Spring Boot microservices, beter gebruik gemaakt kan worden van Fluentd in de plaats van Logstash.